\documentclass{supervision}
\usepackage{course}

\Supervision{1}
\begin{document}
  \begin{questions}
    \section*{Chapter 1}
    \SetQuestionNumber{7}
    \question Suppose the demand curve is $D(p) = 100 - 2p$. What price would
      the monopolist set if he had 60 apartments? How many would he rent? What
      price would he set if he had 40 apartments? How many would he rent?

      \begin{solution}
        $\mbox{Profit } = p \times (100 - 2p) = 100p - 2p^2$.

        By differentiating this it can be determined that the maximal point is
        at $p = 25$ which corresponds to a demand for 50 apartments.

        If he only had 40 apartments then he could increase the price to 30
        which would reduce the demand to 40.
      \end{solution}

    \section*{Chapter 2}
    \SetQuestionNumber{6}
    \question Suppose that a budget equation is given by $p_1 x_1 + p_2 x_2 =
      m$. The government decides to impose a lump-sum tax of $u$, a quantity tax
      on good 1 of $t$, and a quantity subsidy on good 2 of $s$. What is the
      formula for the new budget line?

      \begin{solution}
        $ (p_1 + t) x_1 + (p_2 - s) x_2 = m - u$
      \end{solution}

    \section*{Chapter 3}
    \SetQuestionNumber{8}
    \question Explain why convex preferences means that ``averages are preferred
      to extremes''.

      \begin{solution}
        If the curve is convex then minimizing the sum of the $x$ and $y$ parts
        happens at the middle rather than the extremes, as you deviate from the
        average position a small decrease in one requires a larger increase in
        the other to be as preferable.
      \end{solution}

    \section*{Chapter 5}
    \SetQuestionNumber{3}
    \question Suppose that a consumer always consumes 2 spoons of sugar with
      each cup of coffee. If the price of sugar is $p_1$ per spoonful and the
      price of coffee is $p_2$ per cup and the consumer has $m$ dollars to spend
      on coffee and sugar, how much will he or she want to purchase?

      \begin{solution}
        \begin{align*}
          m &= 2x \times p_1 + x \times p_2 \\
          x &= \frac{m}{2p_1 + p_2}
        \end{align*}
      \end{solution}

    \section*{Chapter 15}
    \SetQuestionNumber{4}
    \question Suppose that the demand curve for a good is given by $D(p) =
      100/p$. What price will maximize revenue?

      \begin{solution}
        Any price, since revenue is demand multiplied by price which is constant
        with this demand curve.
      \end{solution}

    \section*{Chapter 16}
    \SetQuestionNumber{4}
    \question The United States imports about half of its petroleum needs.
      Suppose that the rest of the oil producers are willing to supply as much
      oil as the United States wants at a constant price of \$25 a barrel. What
      would happen to the price of domestic oil if a tax of \$5 a barrel were
      placed on foreign oil?

      \begin{solution}
        It would increase by \$5 as they are direct substitutes.
      \end{solution}


    \section*{Chapter 34}
    \SetQuestionNumber{1}
    \question If the cost to a customer from switching long-distance carriers is
      on the order of \$50, how much should a long-distance carrier be willing
      to pay to acquire a new customer?

      \begin{solution}
        They should pay \$50 so that the customer has no disincentive to move.
      \end{solution}


  \end{questions}
\end{document}
